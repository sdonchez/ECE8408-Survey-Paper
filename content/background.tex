%!TeX root = ../SDonchezECE8408SurveyPaper.tex

\section{background}\label{sec:background}
The below content attempts to provide a brief background on the key characteristics of satellite networks in order to facilitate an informed discussion of routing algorithms in the subsequent sections. It is by no means intended to be a complete or robust discussion of these networks. For a more comprehensive discussion of this material, see \cite{elbert_introduction_2008}.

\subsection{Satellite Orbits}\label{subsec:satelliteOrbits}
At the most fundamental level, satellites are classified based on the altitude at which they orbit the earth. Broadly, there are three categories into which these orbits fall: Low Earth Orbit (LEO), Medium Earth Orbit (MEO), and Geostationary or Geosynchronous Earth Orbit (GEO)\cite{stone_introduction_2004}. Satellites that orbit at an altitude between 100 to 1200 miles are considered to be LEO satellites, while those that orbit at an altitude between 4,000 and 12,000 miles are said to be MEO satellites\cite{stone_introduction_2004}. Meanwhile, GEO satellites orbit at precisely 23,400 miles in altitude, above the equator, and are noteworthy because their orbits are such that they remain "fixed" above a particular spot on the earth\cite{stone_introduction_2004}. 

These classifications become important when one considers multi-layer satellite constellations, such as those discussed in Section \ref{sec:intraLayer}, as these constellations typically are composed of satellites that fall into multiple of the aforementioned categories. As a general rule, constellations at a lower altitude require far more nodes to achieve the same coverage as constellations at higher altitudes\cite{elbert_introduction_2008}. Conversely, constellations at a higher altitude have a much higher propagation delay than those at lower altitudes, directly impacting the speed of data transmission between a ground station and the satellite\cite{elbert_introduction_2008}.

\subsection{Satellite Network Topology}\label{subsec:satelliteNetworkTopology}
It is also important that one understands the basic concepts pertaining to the arrangement of satellites within a network, as this often has direct implications on the routing of data between said nodes. In both LEO and MEO constellations, satellites are typically arranged in a two-dimensional grid (which, of course, is actually wrapped as to be spherical in three dimensional space). In such implementations, one typically considers the grid to be composed of a set of "planes" in one dimension, each of which is comprised of a number of satellites that share a common orbit \cite{xiaogang_survey_2016}. Links between satellites within a given plane (known as intra-plane satellite inter satellite links, or intra-plane ISLs) are generally trivial to construct and maintain, as the distance between adjacent coplanar satellites is roughly fixed. Meanwhile, links between satellites in adjacent planes, or inter-plane ISLs, (which are necessary for any semblance of global or even regional routing), are appreciably more complex on account of the potential for varying distance between satellites or even handover between different coplanar satellites over the course of the two planes' different rotations.

Geostationary Satellites, meanwhile, can only posses an equatorial orbit at a single fixed altitude, meaning that all satellites are in the same plane. Furthermore, access to a "slot" in the geosynchronous orbit is highly coveted and therefore highly expensive. Accordingly most satellites in a GEO orbit are not necessarily interfacing with other GEO satellites in the sense that a MEO or LEO constellation would. These satellites typically instead interact directly with ground stations or, alternatively, with satellites in lower orbits, in which they may act in a supervisory role.

Interactions between orbits, such as those just discussed, compose the other fundamental type of link found in most satellite networks. These links, known as Inter-Orbit Links or IOLs, are often highly dynamic, due to the differing angular speeds of the satellites they are constructed between \cite{xiaogang_survey_2016}. Such links often require an entirely different routing mechanism than those utilized for the ISLs discussed above.