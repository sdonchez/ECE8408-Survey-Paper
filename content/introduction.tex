%!TeX root = ../SDonchezECE8408SurveyPaper.tex
\section{Introduction}\label{sec:introduction}
\IEEEPARstart{R}{outing} algorithms, as they apply to traditional terrestrial networks, particularly of the wired variety, are a well understood and standardized field. Almost all devices on the internet implement either Distance Vector (DV) or Open Shortest Path First (OSPF) algorithms in some form or another. Even in more esoteric use cases, such as in embedded applications with limited compute resources, or in ad-hoc networks lacking a traditional hierarchical structure, most devices implement a variation of the DV algorithm. Although incremental refinement and performance enhancement of these algorithms is (and likely will be for perpetuity) ongoing, and although one can never rule out a sudden transformative revelation in this field, it is likely that these algorithms will continue to be the gold standard that is employed in devices worldwide for the forseeable future.

Much less understood (and less concrete) is the application of routing as it pertains to extra-terrestrial (that is, satellite) networks. Satellite networks form the backbone of a substantial portion of today's communications traffic, facilitating rural broadband, critical civil and military communication, relays for news and sporting broadcasts, and weather and climate observation. As should be no surprise, these networks, on account of their drastically different operating environment, transmission mechanisms, and purpose, impose a unique set of routing requirements that are not satisfied by the traditional terrestrial routing algorithms discussed above. Instead, the unique circumstances surrounding satellite networks require the implementation of a tailored set of routing algorithms, in order to ensure reliability while also maintaining performance and minimizing resource consumption.

One can hardly discuss satellite networks in the modern day without considering the drastic changes brought about by SpaceX's Starlink Platform. The Starlink satellite constellation, which aims to bring high speed internet via satellite into the mainstream market, consists (as of the authoring of this paper) of 1,991 active satellites, with more being launched on an almost monthly (if note more frequent) basis. The company currently plans to deploy over 4,000 satellites, and is actively seeking authorization to expand the constellation to roughly 30,000 satellites in its second generation system. By contrast, most current constellations are comprised of at most double-digit numbers of satellites, with GPS (which is perhaps the most ubiquitous) utilizing only 24 satellites.

Proposed constellations by other technology and aerospace giants provide indications that Starlink is merely the first of a new wave of "megaconstellations" that are in planning stages. Boeing, Amazon, and several other firms are planning similar projects, and more will invariably follow as the technology matures.

Constellations of this size introduce additional complexity into the implementation of satellite routing algorithms, as they result in drastic increases in the number of potential routes that need to be computed. Furthermore, the deployment of satellites at this scale (at least to date) has resulted in much smaller satellites than legacy systems, at a cost of reduced resources such as energy and computing power. These factors, in conjunction with the intended use of the constellations in a high throughput environment with variable loading, make optimizing the routing of data a task of critical priority for system architects and designers.

\subsection{Organization of this Work}\label{subsec:organization}
The remainder of this paper is organized thematically, and covers several categories of satellite routing algorithms. Specifically, Section \ref{sec:background} provides some amount of background information that is relevant to the discussion of routing in satellite networks, such as their composition, hierarchy, and the types of communications links used to connect them. This is critical, as most academics who routinely work with routing algorithms likely do so in the context of terrestrial networks, wherein geographic location is irrelevant (whereas it, along with other characteristics) are critical to the successful implementation of routing in a satellite constellation. 

Section \ref {sec:groundToSpace} discusses algorithms used to provide routing functionality between ground stations and the satellite constellation, as without this link the constellation itself affords no meaningful functionality to users on the ground. Meanwhile, Section \ref{sec:singleLayer} surveys algorithms which propose means of performing routing functionality between satellites in a single layer system (this distinction is elaborated in Section \ref{sec:background}), or, similarly, between co-located satellites in a multi-layer hierarchical constellation. Section \ref{sec:intraLayer} discuss algorithms which facilitate routing between layers of such a multi-layer hierarchical constellation, as this necessarily requires different techniques than are employed in nodes at the same altitude.  

All of these sections will integrate commentary on the applications of these algorithms to scaled up networks such as Starlink, and any other implications such networks have on the state of research in said fields. Finally, Section \ref{sec:conclusion} concludes this paper with a brief summary of the various algorithms presented in the preceding sections, and also highlights emerging work in this field and potential avenues for future research.