%!TeX root = ../SDonchezECE8408SurveyPaper.tex

\section{Inter-Layer Satellite Network Routing}\label{sec:interLayer}
The routing algorithms discussed in Section \ref{sec:singleLayer} discuss solely those links which form between satellites in orbits at approximately the same altitude (traditionally accepted to be LEO). However, many larger networks utilize a multi-layer design, wherein satellites in higher layers serve in a supervisory capacity to those in lower layers. Such an architecture has many advantages as compared to a single layer design, in that the supervisory satellites are able to monitor the status of many more satellites than adjacent nodes in a single-layer design, and therefore can better understand traffic loads in the network. This enables the supervisory satellites to make adjustments to routing in the LEO constellation, greatly improving efficiency. 

Additionally, such a design lends itself to centralization of computing power. Due to their increased altitude and corresponding increased footprint, far fewer MEO or GEO satellites are required than LEO satellites. Accordingly, these satellites can be designed to have more processing and storage resources, enabling cost reduction for the at-scale production of the LEO satellites. A number of algorithms have been proposed that address routing in multi-layer networks based on these advantages. Some key algorithms are discussed below.

\subsection{MLSR Algorithm}\label{subsec:mlsr}
The aptly named Multilayer Satellite Routing (MLSR) Algorithm, proposed in \cite{akyildiz_mlsr_2002}, is widely considered to be a seminal algorithm in this field. In an MLSR system, all data routing occurs exclusively within the LEO layer. Meanwhile, each LEO node reports link delay information up to the MEO satellite assigned as its supervisor, which is used to construct a group level link delay report. The MEO supervisor similarly reports this information up to its GEO supervisor, which performs uses ISLs to disperse this data to the other GEO satellites. Once all GEO nodes have the entire delay situation, they calculate updated routing tables for the groups they manage, and send this information down to the MEO supervisors, who use it to construct and disseminate individual routing tables to their LEO nodes.

It should be noted that this process, although it facilitates online routing with complete situational awareness with regards to link delays across the entire constellation, is not without its fallbacks. Specifically, the periodicity of the system, as well as the delay in reporting up and back down the IOLs, means that routing updates may lag behind the changes that would preempt them in an on-demand network. Additionally, the delay data transmitted is only the propagation delay of a link, as opposed to including the queueing delay, meaning that the algorithm does not afford any congestion management functionality to the constellation.

\subsection{SARA}\label{subsec:sara}
One attempt to resolve the periodicity of these algorithms can be found in the Snapshot-Based Autonomous Routing Algorithm (SARA). In SARA, a VTR based algorithm is implemented in an online mode, with GEO satellites directly managing LEO nodes (without a MEO layer). SARA augments a traditional VTR algorithm by introducing the concept of a Failure Satellite Advertisement (F-SA), which indicates that a node has lost connection. In such an event, an adjacent node transmits a F-SA to the GEO satellite, which triggers a recomputation of the routing tables for the LEO nodes.

\subsection{Load-Balanced Cooperative Data Transmission}\label{subsec:loadBalanced}
While most multi-layer satellite constellations perform data transfer exclusively within the LEO constellation, it is conceivably possible to utilize the MEO satellites as relay stations for data transfer as a means of congestion avoidance or resolution. In \cite{li_load-balanced_2018}, the authors propose the concept of a relay node at the MEO level for each LEO node as well as a manager. Using this second link, the LEO node can elect to bypass the normal ISLs and use a ``bent pipe'' style of communication to transmit directly to the destination LEO node if it is within range of the relay satellite selected by the sending node.