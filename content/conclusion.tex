%!TeX root = ../SDonchezECE8408SurveyPaper.tex

\section{Implications of Megaconstellations}\label{subsec:megaconstellations}
The advent of Starlink, OneWeb, and other proposed megaconstellations has challenged many conventionally held beliefs about the design of such networks. In particular, it was generally presumed until very recently that any sufficiently complex constellation would be a multi-level constellation, but both Starlink and OneWeb are LEO-only networks. This is feasible on account of the ability to provide central coordination, such as would typically be maintained by a MEO or GEO satellite, by a number of ground stations that are all connected in a terrestrial network. In fact, the current generation of Starlink satellites feature no ISLs whatsoever (although plans indicated that the next generation will), but instead function solely in a ``bent pipe'' mode, whereby data from an end user is relayed through one satellite to a ground station in relative geographic proximity\cite{chaudhry_laser_2021}.

That being said, the advent of the next generation of these satellites will definitely pose a challenge from a standpoint of routing. With an anticipated 30,000 satellites in the final constellation, the path between any two nodes becomes many orders of magnitude more complex than has been experienced in historical constellations such as the GPS constellation. Furthermore, one of the key tenants of the Starlink system is that the satellites are small form factor, such that up to 60 of the current generation of the satellites can be launched in a single standard rocket payload. This necessarily constrains power, storage, and compute resources, requiring any on-node portions of the algorithm to be very efficient and lightweight.

\section{Avenues for Future Research}\label{futureResearch}
As was indicated in the preceding sections, the works surveyed herein (and those considered but not ultimately discussed) each seem to focus on a specific aspect of the satellite routing problem. Accordingly, there is a need for a comprehensive solution that attempts to provide optimal routes, while also providing survivability and congestion control. Of course, such an objective is easy to describe in a sentence and many orders of magnitude harder to actually design.

One specific area which seems to be in need of academic attention is that of congestion management. A traditional satellite network has a considerable advantage over most terrestrial networks in that it is well connected. In most terrestrial networks, a star-based topology is implemented, as opposed to the grid featured in a LEO or MEO constellation. This grid structure makes avoiding a congested node a much more feasible task than it is in the internet and similar networks, but requires careful planning so as to not over-react to transient congestion.

Another area that warrants consideration is the concept of relay satellites discussed in Section \ref{subsec:loadBalanced}. This ability to directly avoid large swaths of a constellation by means of a link to a higher level node would provide tremendous flexibility if integrated into megaconstellations, especially as transversal of many nodes such as is required to reach a considerable distance incurs nontrivial delays within each node along the path.

\section{Conclusion}\label{sec:conclusion}
Designing and implementing a routing architecture for a satellite constellation requires consideration of a number of factors that are not traditionally considered when designing a terrestrial network. These factors include the highly dynamic structure of such a constellation, as orbiting nodes frequently vary from each other in distance and may even change neighbors as they orbit. Furthermore, the decentralized nature of most constellations (specifically single-layer networks) makes congestion management difficult, and limited compute and storage resources constrain the computational complexity of any algorithm components that operate on the nodes themselves.

The dawn of megaconstellations such as Starlink have ushered in a new era of satellite networks, placing a renewed focus on the need for efficient routing techniques in such networks. As these types of constellations continue to become more prevalent, such problems will only continue to persist, requiring the design of further algorithms in order to meet the needs of a world connected by such networks.